\documentclass[12pt]{book}
\usepackage[utf8]{inputenc}
\usepackage{amsmath, amssymb}
\usepackage{geometry}
\usepackage{lipsum} % Pour insérer du texte exemple (à remplacer par votre contenu)

\geometry{a4paper}

% Couverture
\title{Résumé du cours d'Anneaux et Modules}
\author{Lounes Boubekir}
\date{\today}

\begin{document}

\maketitle

% Avant propos
\chapter*{Avant propos}

Ce document est un résumé non officiel du cours d'Anneaux et Modules. Il est rédigé à des fins pédagogiques et pourrait contenir des erreurs ou des imprécisions. En aucun cas, il ne remplace le cours original dispensé par votre enseignant.

Nous vous recommandons vivement de vous référer au cours officiel et aux manuels recommandés pour toute clarification.

\vfill
\textit{Bonne lecture !}

\newpage

% Table des matières
\tableofcontents
\newpage

% Chapitres (Exemple de contenu, à remplacer par votre propre résumé)
\chapter{Introduction aux anneaux}
\section{Définition et exemples}
\lipsum[1-2] % Texte exemple
\section{Propriétés des anneaux}
\lipsum[3-4] % Texte exemple

\chapter{Modules sur un anneau}
\section{Définitions et exemples}
\lipsum[5-6] % Texte exemple
\section{Sous-modules et morphismes}
\lipsum[7-8] % Texte exemple

% Ajoutez d'autres chapitres et sections selon vos besoins.

\end{document}
